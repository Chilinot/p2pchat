\section{Introduction}
% Introduction: a summary of what the program does

\section{Compilation \& Usage}
% Use cases: how to compile and run your program, including key examples
\subsection{Compilation}
In order to compile the program you will need the following installed on your system:
\begin{itemize}
  \item Rust: the programming language.
  \item Cargo: the package manager for the Rust language.
\end{itemize}

\noindent Compile the program by issuing the following command: \cmd{cargo build --release}. This will put a binary under the path \cmd{target/release} called \cmd{p2pchat}.

\subsection{Usage}
Issuing the command \cmd{p2pchat -h} displays the following output:
\begin{lstlisting}
Usage:
    p2pchat [OPTIONS] USERNAME PORT

P2P Chat system built in Rust as the final project for the LACPP-course.

positional arguments:
  username              Username to use for the chat.
  port                  Local port for incoming connections.

optional arguments:
  -h,--help             show this help message and exit
  -v,--verbose          Output lots of info.
  -r,--remote REMOTE    Define remote hosts.
  --no-client           Disables the client part of the program.
                        It will not connect to remote hosts.
\end{lstlisting}

\noindent The program supports three commands when started, namely:
\begin{enumerate}
  \item \cmd{connect}: connect to a remote client. Example: \cmd{connect 127.0.0.1:1234}
  \item \cmd{say}: broadcast a message over the network. Example \cmd{say Hi there!}
  \item \cmd{quit}: terminates the program.
\end{enumerate}

\section{Program Documentation}
% Program documentation: a description of important data structures, algorithms, functions
\begin{figure}
  \centering
  \def\svgwidth{\columndwidth}
  \input{figures/p2pchat_structure.pdf_latex}
\end{figure}

When the program is first executed it creates two new actors. One actor that listens for new tcp-connections called the ``ConnectionListener'', and one actor that keeps track of all connected clients and their associated actors. The tracking actor is called the ``ActorManager''. The ``ActorManager'' also is the one responsible for propagating messages between all active connections. When it receives a new message it broadcasts it to all other actors, which in turn send the message to their connected remote clients.

When a new connection is established the ``ConnectionListener'' will spawn two new actors. One listens for new data on the socket, and the other waits for data to send back over the socket. It then notifies the ``ActorManager'' about these two new actors and the connection they represent by sending it a message.

\section{Performance Evaluation}
% Performance evaluation: a discussion of your program’s performance, e.g., observed speedup on multi-core hardware

\section{Concurrency Abstractions}
% Concurrency abstractions: a discussion of the benefits and drawbacks of your chosen concurrency abstraction in the context of your project, especially in comparison to your partner’s abstraction

\section{Discussion}
% Known shortcomings: a description of things that don’t work as well as you would like, or that could be added in the future

% No reliable solution for handling receiving duplicate messages due to triangles or other structures in the network. Routing protocol?

% Broadcasting all incoming messages is a rather naive solution to propagating messages. This could potentially swamp the network in a lot of data every time a new message is sent.
